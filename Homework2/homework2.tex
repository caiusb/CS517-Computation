\documentclass[letterpaper,11pt]{article}

\usepackage[paper=letterpaper,margin=1in]{geometry}


%% minimal example of some latex stuff

\usepackage{xspace,graphicx,amsmath,amsthm,amssymb,xcolor}


\newtheorem{theorem}{Theorem}

%% code boxes

\usepackage{varwidth}
\newcommand{\fcodebox}[1]{%
    \framebox{\codebox{#1}}%
}
\newcommand{\codebox}[1]{%
        \begin{varwidth}{\linewidth}%
        \begin{tabbing}%
            ~~~\=\quad\=\quad\=\quad\=\quad\=\quad\=\kill
            #1
        \end{tabbing}%
        \end{varwidth}%
}

%% complexity classes

\newcommand{\cc}[1]{\ensuremath{\textsf{#1}}\xspace}
\renewcommand{\P}{\cc{P}}
\newcommand{\NP}{\cc{NP}}
\newcommand{\EXP}{\cc{EXP}}
\newcommand{\coNP}{\cc{coNP}}
\renewcommand{\L}{\cc{L}}
\newcommand{\NL}{\cc{NL}}
\newcommand{\PH}{\cc{PH}}
\newcommand{\PSPACE}{\cc{PSPACE}}
\newcommand{\Ppoly}{\cc{P/poly}}
\newcommand{\ZPP}{\cc{ZPP}}
\newcommand{\BPP}{\cc{BPP}}
\newcommand{\RP}{\cc{RP}}
\newcommand{\sharpP}{\cc{$\sharp$P}}
\newcommand{\AM}{\cc{AM}}
\newcommand{\MA}{\cc{MA}}
\newcommand{\IP}{\cc{IP}}


\newcommand{\DTIME}{\cc{DTIME}}
\newcommand{\NTIME}{\cc{NTIME}}
\newcommand{\DSPACE}{\cc{DSPACE}}
\newcommand{\NSPACE}{\cc{NSPACE}}
\newcommand{\SIZE}{\cc{SIZE}}


\newcommand{\karp}{\le_{\textsf{P}}}
\newcommand{\notkarp}{\nleq_{\textsf{P}}}

%% decision problems

\newcommand{\prob}[1]{\ensuremath{\textsc{#1}}\xspace}
\newcommand{\SAT}{\prob{sat}}
\newcommand{\ThreeSAT}{\prob{3sat}}

%% homework templates

\newcounter{problem}
\newcounter{subproblem}[problem]

\newenvironment{problem}%
{%
	\stepcounter{problem}%
	\textbf{\theproblem.}
	\large
}{\\}%

\newenvironment{subproblem}%
{%
	\stepcounter{subproblem}%
	\textbf{\alph{subproblem})}
	\large
}{\\}%

%%%%

\newcommand{\tm}{Turing Machine}
\newcommand{\beq}{\begin{equation}}
\newcommand{\eeq}{\end{equation}}

\title{Homework 2}
\author{Caius Brindescu}

\begin{document}

\maketitle

\begin{problem}
%
Let $A$ and $B$ be decision problems and define
\[
    A \oplus B \overset{\scriptsize\textrm{def}}= 
    \{ x \mid x \mbox{ is in exactly one of } A, B \}
\]
Show that $\NP \cap \coNP$ is closed under the $\oplus$ operation.
\end{problem}

%%%%
\begin{proof}
\end{proof}


\begin{problem}
In this problem DNF refers to disjunctive normal form ({\sc or} of {\sc and} of literals) and CNF refers to conjunctive normal form ({\sc and} of {\sc or} of literals). Define the following:
\begin{align*}
    \textsc{cnf-sat} &= \{ \phi \mid \phi \mbox{ is a satisfiable boolean CNF formula}\} \\
    \textsc{dnf-sat} &= \{ \phi \mid \phi \mbox{ is a satisfiable boolean DNF formula}\} 
\end{align*}
Prove that $\P = \NP$ if and only if $\textsc{cnf-sat} \karp \textsc{dnf-sat}$.

{\em Hint:} in one direction, the Karp reduction is stupid.
\end{problem}

%%%%%%
\begin{proof}
A DNF boolean formula has the following form:
\[
\phi = ( x_{1} \wedge x_{2} \wedge \ldots \wedge x_{n} ) \vee ( \ldots )
\]

I will now try to prove that $\textsc{cnf-sat} \karp \textsc{dnf-sat}$.

Such a formula is satisfiable if \textit{any} of it's clauses is satiasfiable.
A clause is satisfiable iff it does not contain both $x_{i}$ and $\overline{x_{i}}$.
Therefore, find a satisfiable condition can be done in \P{} time.
The algorithm just has too loop through each clause and stop when it finds the first satisfiable one.

Now, I must find a function $f$ that can convert any $\textsc{CNF}$ formula into an equivalent $\textsc{DNF}$ in \P{} time. 
Such an $f$ does not exist \cite{dnf}.

Therefore, $\textsc{DNF-SAT} \notkarp \textsc{CNF-SAT}$.
This means that $\P \neq \NP$.

Similarly, proving in the other direction, if $\P = \NP$, then function $f$ would exist, and the reduction would be possible.
But it doesn't so it isn't.
So $\P \neq \NP$.

\end{proof}


\begin{problem}
Show that the following problem is \NP-complete:
\[
    \{ \phi \mid \phi \mbox{ is a satisfiable boolean formula in which no variable appears more than 5 times}\}
\]
\end{problem}

%%%%%%%%
\begin{proof}
I will name the problem as $\textsc{5var-SAT}$.
I will try to prove that $\textsc{SAT} \karp \textsc{5var-SAT}$.

I can use the following algorithm to convert and \textsc{SAT} problem to a \textsc{5var-SAT} problem in $\P$ time.
For each variable that occurs more than five times, each occurance that happens after the fifth will be replaced by a new variable.
I do this until I have an expression where each variable occurs at most five times.

The algorithm runs in $P$ time, since the output is of the same size as the input.
Also, in the worst case, I will only create $|v| / 5$ new variables, where $|v|$ is the number of input variables.

Since every $\textsc{SAT}$ problem can be reduced to a $\textsc{5var-SAT}$ problem, $\textsc{5var-SAT}$ is $\NP$-complete.
\end{proof}

\begin{problem}
Show that the following problem is \NP-complete: given a set of linear inequalities over a set of variables $x_1, \ldots, x_n$, determine whether there is an assignment of {\em integers} to the $x_i$'s that satisfies all inequalities.

{\em Example:}
\begin{align*}
    x_1 + 3x_2 - x_5 &\ge 3 \\
    2x_2 - x_4 &\le x_3 \\
    x_1 + x_2 + x_3 + x_4 + x_5 &\ge 0
\end{align*}
is satisfied by $x_1 = 10$; $x_2 = x_3 = x_4 = x_5 = 0$ (among many others).
\end{problem}

%%%%%%%%%%
\begin{proof}
I will show that $\textsc{3SAT} \karp \textsc{LIS}$, where $\textsc{LIS}$ stands for {\it Linear Inequality Set}.
In this proof I have used material from~\cite{clasical}.

To do this, I will transform a 3-{\sc CNF} formula into an {\sc LIS}.
Each clause will become a new inequality in the set.
In each clause, I will replace each term $k_{i}$ with a new integer variable.
For each negated term $\overline{k+{i}}$, I will replace it with $1-k{i}$.
Each clause will be less than or equal to 0.
Therefore, for the clause $(x_{1} \vee x_{2} \vee \overline{x_{3}}) \wedge (x_{2} \vee \overline{x_{4}} \vee x_{5})$ I will have the following system:
%
\begin{align*}
	k_{1} + k_{2} - k_{3} &\ge 1 \\
	k_{2} - k_{4} - k_{5} &\ge 2 
\end{align*}

If the above system allows a solution, then so should the 3-{\sc CNF} formula.
This means that $\textsc{3SAT} \karp \textsc{LIS}$.
Therefore, I conclude that {\sc LIS} is an $\NP$-complete problem.

\end{proof}

\begin{problem}
A 3-coloring of a graph is a way to color each vertex red, green, or blue so that no edge in the graph has endpoints with the same color. Determining whether a graph is 3-colorable is \NP-complete.

Suppose you have a subroutine that tells you whether the given graph has a 3-coloring (i.e., it responds {\sc yes} or {\sc no}). Show how to use the subroutine to compute a 3-coloring of any 3-colorable graph (i.e., output a legal assignment of colors to the vertices).
\end{problem}

%%%%%%%%%%%
\begin{proof}
If a graph is 3-colorable then I start by assigning a color to a vertex.
I will then ask the subroutine to check if the graph is still colorable with the given constraint.
If it isn't, I will assign a new color and check again.
After 3 checks, I will have to get a {\sc YES} answer.
The graph is 3-colorable, therefore there must exist a valid assignment.
I will do this for every vertex until the graph is colored.

Assuming that the checking subroutine can done in constant time (it isn't but I will assume anyway to simply the explanation), then the coloring is done in linear time.
I will have to assign, at most, $3*|V|$ colors, where $|V|$ is the number of vertices.
\end{proof}

\begin{thebibliography}{1}

\bibitem{dnf} Miltersen P. et al. 2003. {\em On converting CNF to DNF.} Electronic Colloquium on Computational Complexity, Report No. 17.

\bibitem{clasical} Kitaev, A. Yu. et al. 2002. {\em Classical and Quantum Computation}. American Mathematical Society, Boston, MA, USA.


\end{thebibliography}

\end{document}