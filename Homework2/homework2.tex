\documentclass[letterpaper,11pt]{article}

\usepackage[paper=letterpaper,margin=1in]{geometry}


%% minimal example of some latex stuff

\usepackage{xspace,graphicx,amsmath,amsthm,amssymb,xcolor}


\newtheorem{theorem}{Theorem}

%% code boxes

\usepackage{varwidth}
\newcommand{\fcodebox}[1]{%
    \framebox{\codebox{#1}}%
}
\newcommand{\codebox}[1]{%
        \begin{varwidth}{\linewidth}%
        \begin{tabbing}%
            ~~~\=\quad\=\quad\=\quad\=\quad\=\quad\=\kill
            #1
        \end{tabbing}%
        \end{varwidth}%
}

%% complexity classes

\newcommand{\cc}[1]{\ensuremath{\textsf{#1}}\xspace}
\renewcommand{\P}{\cc{P}}
\newcommand{\NP}{\cc{NP}}
\newcommand{\EXP}{\cc{EXP}}
\newcommand{\coNP}{\cc{coNP}}
\renewcommand{\L}{\cc{L}}
\newcommand{\NL}{\cc{NL}}
\newcommand{\PH}{\cc{PH}}
\newcommand{\PSPACE}{\cc{PSPACE}}
\newcommand{\Ppoly}{\cc{P/poly}}
\newcommand{\ZPP}{\cc{ZPP}}
\newcommand{\BPP}{\cc{BPP}}
\newcommand{\RP}{\cc{RP}}
\newcommand{\sharpP}{\cc{$\sharp$P}}
\newcommand{\AM}{\cc{AM}}
\newcommand{\MA}{\cc{MA}}
\newcommand{\IP}{\cc{IP}}


\newcommand{\DTIME}{\cc{DTIME}}
\newcommand{\NTIME}{\cc{NTIME}}
\newcommand{\DSPACE}{\cc{DSPACE}}
\newcommand{\NSPACE}{\cc{NSPACE}}
\newcommand{\SIZE}{\cc{SIZE}}


\newcommand{\karp}{\le_{\textsf{P}}}

%% decision problems

\newcommand{\prob}[1]{\ensuremath{\textsc{#1}}\xspace}
\newcommand{\SAT}{\prob{sat}}
\newcommand{\ThreeSAT}{\prob{3sat}}

%% homework templates

\newcounter{problem}
\newcounter{subproblem}[problem]

\newenvironment{problem}%
{%
	\stepcounter{problem}%
	\textbf{\theproblem.}
	\large
}{\\}%

\newenvironment{subproblem}%
{%
	\stepcounter{subproblem}%
	\textbf{\alph{subproblem})}
	\large
}{\\}%

%%%%

\newcommand{\tm}{Turing Machine}
\newcommand{\beq}{\begin{equation}}
\newcommand{\eeq}{\end{equation}}

\title{Homework 2}
\author{Caius Brindescu}

\begin{document}

\maketitle

\begin{problem}
%
Let $A$ and $B$ be decision problems and define
\[
    A \oplus B \overset{\scriptsize\textrm{def}}= 
    \{ x \mid x \mbox{ is in exactly one of } A, B \}
\]
Show that $\NP \cap \coNP$ is closed under the $\oplus$ operation.
\end{problem}

%%%%
\begin{proof}
\end{proof}


\begin{problem}
In this problem DNF refers to disjunctive normal form ({\sc or} of {\sc and} of literals) and CNF refers to conjunctive normal form ({\sc and} of {\sc or} of literals). Define the following:
\begin{align*}
    \textsc{cnf-sat} &= \{ \phi \mid \phi \mbox{ is a satisfiable boolean CNF formula}\} \\
    \textsc{dnf-sat} &= \{ \phi \mid \phi \mbox{ is a satisfiable boolean DNF formula}\} 
\end{align*}
Prove that $\P = \NP$ if and only if $\textsc{cnf-sat} \karp \textsc{dnf-sat}$.

{\em Hint:} in one direction, the Karp reduction is stupid.
\end{problem}

%%%%%%


\begin{problem}
Show that the following problem is \NP-complete:
\[
    \{ \phi \mid \phi \mbox{ is a satisfiable boolean formula in which no variable appears more than 5 times}\}
\]
\end{problem}

%%%%%%%%

\begin{problem}
Show that the following problem is \NP-complete: given a set of linear inequalities over a set of variables $x_1, \ldots, x_n$, determine whether there is an assignment of {\em integers} to the $x_i$'s that satisfies all inequalities.

{\em Example:}
\begin{align*}
    x_1 + 3x_2 - x_5 &\ge 3 \\
    2x_2 - x_4 &\le x_3 \\
    x_1 + x_2 + x_3 + x_4 + x_5 &\ge 0
\end{align*}
is satisfied by $x_1 = 10$; $x_2 = x_3 = x_4 = x_5 = 0$ (among many others).
\end{problem}

%%%%%%%%%%

\begin{problem}
A 3-coloring of a graph is a way to color each vertex red, green, or blue so that no edge in the graph has endpoints with the same color. Determining whether a graph is 3-colorable is \NP-complete.

Suppose you have a subroutine that tells you whether the given graph has a 3-coloring (i.e., it responds {\sc yes} or {\sc no}). Show how to use the subroutine to compute a 3-coloring of any 3-colorable graph (i.e., output a legal assignment of colors to the vertices).
\end{problem}

\end{document}