\documentclass[letterpaper,11pt]{article}

\usepackage[paper=letterpaper,margin=1in]{geometry}


%% minimal example of some latex stuff

\usepackage{xspace,graphicx,amsmath,amsthm,amssymb,xcolor}


\newtheorem{theorem}{Theorem}

%% code boxes

\usepackage{varwidth}
\newcommand{\fcodebox}[1]{%
    \framebox{\codebox{#1}}%
}
\newcommand{\codebox}[1]{%
        \begin{varwidth}{\linewidth}%
        \begin{tabbing}%
            ~~~\=\quad\=\quad\=\quad\=\quad\=\quad\=\kill
            #1
        \end{tabbing}%
        \end{varwidth}%
}

%% complexity classes

\newcommand{\cc}[1]{\ensuremath{\textsf{#1}}\xspace}
\renewcommand{\P}{\cc{P}}
\newcommand{\NP}{\cc{NP}}
\newcommand{\EXP}{\cc{EXP}}
\newcommand{\NEXP}{\cc{NEXP}}
\newcommand{\coNP}{\cc{coNP}}
\renewcommand{\L}{\cc{L}}
\newcommand{\NL}{\cc{NL}}
\newcommand{\PH}{\cc{PH}}
\newcommand{\PSPACE}{\cc{PSPACE}}
\newcommand{\Ppoly}{\cc{P/poly}}
\newcommand{\ZPP}{\cc{ZPP}}
\newcommand{\BPP}{\cc{BPP}}
\newcommand{\RP}{\cc{RP}}
\newcommand{\sharpP}{\cc{$\sharp$P}}
\newcommand{\AM}{\cc{AM}}
\newcommand{\MA}{\cc{MA}}
\newcommand{\IP}{\cc{IP}}


\newcommand{\DTIME}{\cc{DTIME}}
\newcommand{\NTIME}{\cc{NTIME}}
\newcommand{\DSPACE}{\cc{DSPACE}}
\newcommand{\NSPACE}{\cc{NSPACE}}
\newcommand{\SIZE}{\cc{SIZE}}


\newcommand{\karp}{\le_{\textsf{P}}}

%% decision problems

\newcommand{\prob}[1]{\ensuremath{\textsc{#1}}\xspace}
\newcommand{\SAT}{\prob{sat}}
\newcommand{\ThreeSAT}{\prob{3sat}}

%% homework templates

\newcounter{problem}
\newcounter{subproblem}[problem]

\newenvironment{problem}%
{%
	\stepcounter{problem}%
	\textbf{\theproblem.}
	\large
}{\normalsize~\\}%

\newenvironment{subproblem}%
{%
	\stepcounter{subproblem}%
	\textbf{\alph{subproblem})}
	\large
}{\normalsize~\\}%

%%%%

\newcommand{\tm}{Turing Machine}
\newcommand{\beq}{\begin{equation}}
\newcommand{\eeq}{\end{equation}}

%%%%

\title{Homework 3}
\author{Caius Brindescu}

\begin{document}

\maketitle

\begin{problem}
Prove that $L$ is $\NP$-complete if and only if $\overline L$ is $\coNP$-complete (both notions of completeness under Karp reductions). Conclude that the following problem is $\coNP$-complete:
\[
    \mbox{EQV} = \{ \langle \phi_1, \phi_2 \rangle \mid \phi_1 \mbox{ and } \phi_2 \mbox{ are logically equivalent boolean formulas} \}
\]
\end{problem}

%%%%%%%%%
\begin{proof}

We know by definition that if $L$ is in $\NP$ then $\overline L$ is in $\coNP$.
If $L$ is $\NP$-complete, that means that $\forall A \in \NP, A \karp L$.
This can be further expanded as $\forall A \in \NP, \exists f : x \in A \iff f(x) \in L$.
This can be rewritten as $\forall A \in \NP, \exists f : x \notin A \iff f(x) \notin L$.
Which is equivalent to $\forall A \in \NP, \exists f : x \in {\overline A} \iff f(x) \in {\overline L}$.
However, $\overline A$ is a $\coNP$ problem.
This means that $\forall {\overline A} \in \coNP, \exists f:x \in {\overline A} \iff f(x) \in {\overline L}$.
This can be written ${\overline A} \karp {\overline L}$, $\forall {\overline A} \in \coNP$.
Which means that $\overline L$ is $\coNP$-complete.

Using this, I will prove that $EQV$ is $\coNP$-complete by proving that $\overline EQV$ is $\NP$-complete.
I will define the complement of $EQV$ as:
\[
	\overline{EQV} = \{ \langle \phi_1, \phi_2 \rangle \mid \exists x : \phi_1 (x) \neq \phi_2 (x) \}
\]
I can rewrite $\phi_1 (x) \neq \phi_2 (x)$ as $\phi = \phi_1 (x) \wedge \overline{\phi_2 (x) }=true$.
This transforms $\overline{EQV}$ into a $\SAT$ problem, which I know is $\NP$-complete.
Therefore $\SAT \karp \overline{EQV}$, which means that $\overline{EQV}$ is $\NP$-complete.
Using the proof from the beginning of the exercise, I conclude that $EQV$ is $\coNP$-complete.
\end{proof}

\newcommand{\DNP}{\cc{DNP}}
\begin{problem}
Define the complexity class
$
    \cc{DNP} \overset{\text{def}}= \{ L \mid \exists L_1 \in \NP, L_2 \in \coNP : L = L_1 \cap L_2 \}
$
\end{problem}

\begin{subproblem}
Show that $\NP \ne \coNP \iff \cc{DNP} \ne \NP \cap \coNP$.
\end{subproblem}

\begin{proof}
I will begin by showing the left side ($\Leftarrow$) of the statement.
This means proving that $\DNP \ne \NP \cap \coNP \Rightarrow \NP \ne \coNP$.
I will prove the complement of that statement which is $\NP = \coNP \Rightarrow \DNP = \NP \cap \coNP$.
If $\NP = \coNP$, this means that both $L_1$ and $L_2$ are in the same set. Which means that $L = L_1 = L_2$.
Therefore $\DNP = \NP = \coNP = \NP \cap \coNP$.

Next, I will prove the right side ($\Rightarrow$) of the statement: $\NP \ne \coNP \Rightarrow \DNP \ne \NP \cap \coNP$.
If $\NP \ne \coNP$ this means that $S = \NP \setminus \coNP \ne \emptyset$. 
I choose $L_1 \in S$.
This will allow me to construct $L^* \in \DNP$ and $L^* \notin \NP \cap \coNP$.
Therefore $\DNP \ne \NP \cap \coNP$.
\end{proof}

\begin{subproblem}
Show that the following problem is $\cc{DNP}$-complete under Karp reductions:
    \[
        \mbox{SATUNSAT} = \{ \langle \phi_1, \phi_2 \rangle \mid
            \phi_1 \mbox{ is satisfiable and } \phi_2 \mbox{ is not satisfiable} \}
    \]
\end{subproblem}

%%%%%%%%%%
\begin{proof}
\newcommand{\SATUNSAT}{\mbox{SATUNSAT}}
I need to prove that $\forall A \in \DNP, A \karp \SATUNSAT$.

The $\SAT$ problem $\SAT = \{ \phi \mid \phi \mbox{ is satisfiable }\}$ is $\NP$-complete.
Similarly, the UNSAT problem $\mbox{UNSAT} = \{ \phi \mid \phi \mbox{ is not satisfiable }\}$ is $\coNP$-complete.
This shows that $\SATUNSAT \in \DNP$.

Now, I need to prove that $\SATUNSAT$ is $\DNP$-hard.
Every problem in $\DNP$ can, by definition, be expressed as combination of a $\NP$-problem and $\coNP$-problem.
Since $\SAT$ is $\NP$-hard and UNSAT is $\coNP$-hard (by being the complement of $\SAT$, see Problem 1), that means that every problem in $\DNP$ can be karp-reduced to $\SATUNSAT$.
This proves that $\SATUNSAT$ is $\DNP$-hard.

Therefore, I conclude that $\SATUNSAT$ is $\DNP$-complete.
\end{proof}

\begin{problem}
Show that $\DTIME(2^n) \ne \NTIME(2^n)$ implies  $\DTIME(n) \ne \NTIME(n)$

{\em Hint:} show the contrapositive. Assuming $\DTIME(n) = \NTIME(n)$, use this fact to solve every $\NTIME(2^n)$ problem deterministically in $\DTIME(2^n)$. Take an arbitrary $L \in \NTIME(2^n)$ and define
\[
    L_{\text{pad}} = \{ x\#^t \mid t = 2^{|x|}-|x| \mbox{ and } x \in L \}
\]
where $\#$ is a new symbol not appearing in $x$. What is the computational complexity of $L_{\text{pad}}$? How does this help?
\end{problem}

%%%%%%%%%%%
\begin{proof}
\end{proof}

\begin{problem}
Recall that:
\[
    \EXP = \bigcup_{c>0} \DTIME(2^{n^c}); \qquad\qquad
    \NEXP = \bigcup_{c>0} \NTIME(2^{n^c})
\]
Show that $\EXP \ne \NEXP$ implies $\P \ne \NP$.

{\em Warning:} $\P = \NP$ does not necessarily mean that $\DTIME(n^c) = \NTIME(n^c)$ for every $c$.
\end{problem}

%%%%%%%%%%%
\begin{proof}
\end{proof}

\end{document}