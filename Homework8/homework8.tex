\documentclass[letterpaper,11pt]{article}

\usepackage[paper=letterpaper,margin=1in]{geometry}


%% minimal example of some latex stuff

\usepackage{xspace,graphicx,amsmath,amsthm,amssymb,xcolor}


\newtheorem{theorem}{Theorem}

%% code boxes

\usepackage{varwidth}
\newcommand{\fcodebox}[1]{%
    \framebox{\codebox{#1}}%
}
\newcommand{\codebox}[1]{%
        \begin{varwidth}{\linewidth}%
        \begin{tabbing}%
            ~~~\=\quad\=\quad\=\quad\=\quad\=\quad\=\kill
            #1
        \end{tabbing}%
        \end{varwidth}%
}

%% complexity classes

\newcommand{\cc}[1]{\ensuremath{\textsf{#1}}\xspace}
\renewcommand{\P}{\cc{P}}
\newcommand{\NP}{\cc{NP}}
\newcommand{\EXP}{\cc{EXP}}
\newcommand{\coNP}{\cc{coNP}}
\renewcommand{\L}{\cc{L}}
\newcommand{\NL}{\cc{NL}}
\newcommand{\PH}{\cc{PH}}
\newcommand{\PSPACE}{\cc{PSPACE}}
\newcommand{\Ppoly}{\cc{P/poly}}
\newcommand{\ZPP}{\cc{ZPP}}
\newcommand{\BPP}{\cc{BPP}}
\newcommand{\RP}{\cc{RP}}
\newcommand{\sharpP}{\cc{$\sharp$P}}
\newcommand{\AM}{\cc{AM}}
\newcommand{\MA}{\cc{MA}}
\newcommand{\IP}{\cc{IP}}


\newcommand{\DTIME}{\cc{DTIME}}
\newcommand{\NTIME}{\cc{NTIME}}
\newcommand{\DSPACE}{\cc{DSPACE}}
\newcommand{\NSPACE}{\cc{NSPACE}}
\newcommand{\SIZE}{\cc{SIZE}}


\newcommand{\karp}{\le_{\textsf{P}}}
\newcommand{\randreduc}{\le_{\textsf{rand}}}
\newcommand{\ThreeSAT}{\ensuremath{\textsc{3sat}}\xspace}


%% decision problems

\newcommand{\prob}[1]{\ensuremath{\textsc{#1}}\xspace}
\newcommand{\SAT}{\prob{sat}}

%% homework templates

\newcounter{problem}
\newcounter{subproblem}[problem]

\newenvironment{problem}%
{%
	\stepcounter{problem}%
	\textbf{\theproblem.}
	\large
}{\\}%

\newenvironment{subproblem}%
{%
	\stepcounter{subproblem}%
	\textbf{\alph{subproblem})}
	\large
}{\\}%

%%%%

\newcommand{\tm}{Turing Machine}
\newcommand{\beq}{\begin{equation}}
\newcommand{\eeq}{\end{equation}}

\newcommand{\NPpoly}{\NP/\textsf{poly}}
\newcommand{\coNPpoly}{\coNP/\textsf{poly}}

%%%%

\title{Homework 8}
\author{Caius Brindescu}

\begin{document}

\maketitle

\begin{problem}
Suppose $L$ has an $\AM$ proof with no error; that is, if $x \not\in L$ then Arthur accepts with probability zero. 
Show that $L \in \NP$.
\end{problem}

\begin{proof}

If $L \in \NP$ then $L = \{ x | \exists w : M(x,w) = 1\}$, where $M$ is a deterministic \tm{} that runs in $\P$ time.

Since L has an $\AM$ proof with no error, it can written as:
\begin{align*}
	x \in L &\Rightarrow \exists \pi \forall r : M(x,\pi,r) = 1 \\
	x \not \in L &\Rightarrow \forall \pi : \underset{r}{\Pr}[M(x,\pi,r) = 1] = 0;
\end{align*}

This can be rewritten as:
\[
	x \not \in L \Rightarrow \forall \pi \nexists r: M(x,\pi,r) = 1
\]

This means that there is no random string $r$ that Arthur can send such that Merlin can falsely convince him that $x \in L$.
That means that Merlin will never be able to convince Arthur if $x \not \in L$.

The definition of $x \in L$ is same as the definition of $\NP$ language.
That means, that for any random string $r$ that Arthur sends, Merlin can find a witness, $\pi$, that can be verified in $\P$ time.
Therefore, any problem in $\NP$ can be expressed using an $\AM$ proof with zero error, where Merlin simply gives Arthur a witness.

\end{proof}

%%%%%%%%%%%%

\begin{problem}
\AM is often refered to as a ``randomized version of \NP,'' and in this problem we'll see why.

Define a randomized reduction ``$\randreduc$'' as follows: we say $A \randreduc B$ if there exists a probabilistic polynomial-time TM $M$ such that
    \begin{align*}
        x \in A &\implies \Pr[ M(x) \in B ] \ge 2/3 ;   \\
        x \not\in A &\implies \Pr[ M(x) \in B ] \le 1/3 .
    \end{align*}
The probability is over random coins of $M$.
Prove that $\AM = \{ L \mid L \randreduc \ThreeSAT \}$.
\end{problem}

\begin{proof}
\end{proof}

%%%%%%%%%%%%%

\begin{problem}
Prove that $\AM \subseteq \NP/\textsf{poly} \cap \coNP/\textsf{poly}$. 

So that's two parts: $\AM \subseteq \NP/\textsf{poly}$ and $\AM \subseteq \coNP/\textsf{poly}$. 
Note that $\NP/\textsf{poly}$ is the class of languages decided by NPTMs with polynomial advice. 
\end{problem}

\begin{proof}
I will first prove that $\AM \subseteq \NPpoly$.

First, I will {\bf arithmetize} the boolean formula $\psi$, and get a polynomial $P_\psi$.

The $\textsc{3CNF}$ formula will be converted by replacing all the or's ($\vee$) with addition ($+$) and all the and's ($\wedge$) with multiplication ($\cdot$). 
Also, the boolean variable $x_i$ will be replaced with an integer variable $X_i$, and negation is written as $1-X_i$.

The total degree of the polynomial $P_\psi$ is the total number of clauses in $\psi$.
I will use $m$ to denote the total number of clauses, and, therefore, the degree of $P_\psi$.

$\psi$ satisfiable $\Leftrightarrow \exists$ an assignment $\vec{X}=\{X_1,X_2,...\}$ for $P_\psi$ such that $P_\psi(X) \neq 0$.

Next, I will perform a {\bf multivariate polynomial test}.
\end{proof}

\end{document}