\documentclass[letterpaper,11pt]{article}

\usepackage[paper=letterpaper,margin=1in]{geometry}


%% minimal example of some latex stuff

\usepackage{xspace,graphicx,amsmath,amsthm,amssymb,xcolor}


\newtheorem{theorem}{Theorem}

%% code boxes

\usepackage{varwidth}
\newcommand{\fcodebox}[1]{%
    \framebox{\codebox{#1}}%
}
\newcommand{\codebox}[1]{%
        \begin{varwidth}{\linewidth}%
        \begin{tabbing}%
            ~~~\=\quad\=\quad\=\quad\=\quad\=\quad\=\kill
            #1
        \end{tabbing}%
        \end{varwidth}%
}

%% complexity classes

\newcommand{\cc}[1]{\ensuremath{\textsf{#1}}\xspace}
\renewcommand{\P}{\cc{P}}
\newcommand{\NP}{\cc{NP}}
\newcommand{\EXP}{\cc{EXP}}
\newcommand{\coNP}{\cc{coNP}}
\renewcommand{\L}{\cc{L}}
\newcommand{\NL}{\cc{NL}}
\newcommand{\PH}{\cc{PH}}
\newcommand{\PSPACE}{\cc{PSPACE}}
\newcommand{\Ppoly}{\cc{P/poly}}
\newcommand{\ZPP}{\cc{ZPP}}
\newcommand{\BPP}{\cc{BPP}}
\newcommand{\RP}{\cc{RP}}
\newcommand{\PP}{\cc{PP}}
\newcommand{\ePP}{\cc{ePP}}
\newcommand{\sharpP}{\cc{$\sharp$P}}
\newcommand{\AM}{\cc{AM}}
\newcommand{\MA}{\cc{MA}}
\newcommand{\IP}{\cc{IP}}


\newcommand{\DTIME}{\cc{DTIME}}
\newcommand{\NTIME}{\cc{NTIME}}
\newcommand{\DSPACE}{\cc{DSPACE}}
\newcommand{\NSPACE}{\cc{NSPACE}}
\newcommand{\SIZE}{\cc{SIZE}}


\newcommand{\karp}{\le_{\textsf{P}}}
\newcommand{\Prob}{\mbox{Pr}}
\newcommand{\Time}[1]{\mbox{time}#1}

%% decision problems

\newcommand{\prob}[1]{\ensuremath{\textsc{#1}}\xspace}
\newcommand{\SAT}{\prob{sat}}
\newcommand{\ThreeSAT}{\prob{3sat}}

%% homework templates

\newcounter{problem}
\newcounter{subproblem}[problem]

\newenvironment{problem}%
{%
	\stepcounter{problem}%
	\textbf{\theproblem.}
	\large
}{\\}%

\newenvironment{subproblem}%
{%
	\stepcounter{subproblem}%
	\textbf{\alph{subproblem})}
	\large
}{\\}%

%%%%

\newcommand{\tm}{Turing Machine}
\newcommand{\beq}{\begin{equation}}
\newcommand{\eeq}{\end{equation}}

%%%%

\title{Homework 7}
\author{Caius Brindescu}

\begin{document}

\maketitle

\begin{problem}
$L \in \PP$ iff there exists a PPT $M$ so that $L = \{ x \mid \Prob[M(x) = 1] > 1/2\}$.
Show that $\NP \subseteq \PP$.
\end{problem}

\begin{proof}
I will use a proof similar to the one used to prove that $\BPP \subseteq \Ppoly$.

An $\NP$ problem is defined as $\{x \mid \exists w : M_P(x,w) = 1\}$, where $M_P$ is a deterministic \tm{} that runs in polynomial time.
If $x \in L$, then $\exists r : M(x,r) = 1$
From the definition, I know that the probability of finding r is greater than $1/2$.
Therefore, $r$ can serve as my witness in the $\NP$ problem.

\end{proof}
%%%%%%%

\begin{problem}
Let M be a probabilistic TM, and let $\Time(M,x)$ denote the number of steps taken by $M$ on input $x$.
Not that $\Time(M,x)$ is a {\it random variable} as well.

A probabilistic TM $M$ runs in {\bf strict polynomial-time} if there is a polynomial $p$ such that for all $x$, $\Prob[\Time(M,x) \leq p(|x|)] =1$. 
I.e., $M$ {\it always} takes at most $p(|x|)$ steps.

A probabilistic TM $M$ runs in {\bf expected polynomial-time} if there is a polynomial $p$ such that for all $x$, $\mathbb{E}[\Time(M,x)] \leq p(|x|)$.

The standard complexity classes $\PP$, $\BPP$, etc., are defined in terms of {\bf strict} polynomial time.

Define $\ePP$, to be the same as $\PP$, except we allow the \tm{} to be {\bf expected}-polynomial time.
That is, $L \in \ePP$ if there exists an expected-poly-time probabilistic TM $M$ with $L = \{ x \mid \Prob[M(x) = 1] > 1/2\}$.

Show that $\EXP \subseteq \ePP$.
In fact, it is possible to show that $\EXP$ can be decided (in the $\PP$ sense) by expected-{\it constant-time} probabilistic TMs.
\end{problem}

\begin{problem}
Let $L$ be a problem in $\RP$, so that:
\begin{align*}
	x \in L &\Rightarrow \Prob[M(x) = 1] \geq 1/2 \\
	x \not \in L &\Rightarrow \Prob[M(x) = 0] = 1
\end{align*}
Show that for each $n$, there exists a set $R$ of random tapes for $M$ with $|R| = n$ and satisfying the following property:
\[
	x \in L \Leftrightarrow \exists r \in R : M(x;r) = 1
\]
Show how this gives an alternate proof that $\RP \subseteq \Ppoly$.
\end{problem}
\end{document}