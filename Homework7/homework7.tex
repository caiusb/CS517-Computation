\documentclass[letterpaper,11pt]{article}

\usepackage[paper=letterpaper,margin=1in]{geometry}


%% minimal example of some latex stuff

\usepackage{xspace,graphicx,amsmath,amsthm,amssymb,xcolor}


\newtheorem{theorem}{Theorem}

%% code boxes

\usepackage{varwidth}
\newcommand{\fcodebox}[1]{%
    \framebox{\codebox{#1}}%
}
\newcommand{\codebox}[1]{%
        \begin{varwidth}{\linewidth}%
        \begin{tabbing}%
            ~~~\=\quad\=\quad\=\quad\=\quad\=\quad\=\kill
            #1
        \end{tabbing}%
        \end{varwidth}%
}

%% complexity classes

\newcommand{\cc}[1]{\ensuremath{\textsf{#1}}\xspace}
\renewcommand{\P}{\cc{P}}
\newcommand{\NP}{\cc{NP}}
\newcommand{\EXP}{\cc{EXP}}
\newcommand{\coNP}{\cc{coNP}}
\renewcommand{\L}{\cc{L}}
\newcommand{\NL}{\cc{NL}}
\newcommand{\PH}{\cc{PH}}
\newcommand{\PSPACE}{\cc{PSPACE}}
\newcommand{\Ppoly}{\cc{P/poly}}
\newcommand{\ZPP}{\cc{ZPP}}
\newcommand{\BPP}{\cc{BPP}}
\newcommand{\RP}{\cc{RP}}
\newcommand{\PP}{\cc{PP}}
\newcommand{\ePP}{\cc{ePP}}
\newcommand{\sharpP}{\cc{$\sharp$P}}
\newcommand{\AM}{\cc{AM}}
\newcommand{\MA}{\cc{MA}}
\newcommand{\IP}{\cc{IP}}


\newcommand{\DTIME}{\cc{DTIME}}
\newcommand{\NTIME}{\cc{NTIME}}
\newcommand{\DSPACE}{\cc{DSPACE}}
\newcommand{\NSPACE}{\cc{NSPACE}}
\newcommand{\SIZE}{\cc{SIZE}}


\newcommand{\karp}{\le_{\textsf{P}}}
\newcommand{\Prob}{\mbox{Pr}}
\newcommand{\Time}[1]{\mbox{time}#1}

%% decision problems

\newcommand{\prob}[1]{\ensuremath{\textsc{#1}}\xspace}
\newcommand{\SAT}{\prob{sat}}
\newcommand{\ThreeSAT}{\prob{3sat}}

%% homework templates

\newcounter{problem}
\newcounter{subproblem}[problem]

\newenvironment{problem}%
{%
	\stepcounter{problem}%
	\textbf{\theproblem.}
	\large
}{\\}%

\newenvironment{subproblem}%
{%
	\stepcounter{subproblem}%
	\textbf{\alph{subproblem})}
	\large
}{\\}%

%%%%

\newcommand{\tm}{Turing Machine}
\newcommand{\beq}{\begin{equation}}
\newcommand{\eeq}{\end{equation}}

%%%%

\title{Homework 7}
\author{Caius Brindescu}

\begin{document}

\maketitle

\begin{problem}
$L \in \PP$ iff there exists a PPT $M$ so that $L = \{ x \mid \Prob[M(x) = 1] > 1/2\}$.
Show that $\NP \subseteq \PP$.
\end{problem}

\begin{proof}
To show this I will prove that $\ThreeSAT$, can be solved using a $\PP$ \tm{}.
Since $\ThreeSAT$ is $\NP$-complete, then any problem in $\NP$ can be solved using a $\PP$ machine.
Consider the following PPT \tm{}:

\fcodebox{
\underline{on input $(x,r)$:} \\
\> check if $r$ satisfies $x$ \\
\> if $yes$, accept \\
\> else, accept with probability $1/2$\footnote{$r$ can contain an extra bit to determine if the machine should accept or not}.
}

If $x$ is not satisfiable, then $\Prob[M(x) = 1] = 1/2$.
If $x$ is satisfiable, then $\Prob[M(x) = 1] > 1/2$.
I find a random satisfiable assignment, the machine will accept with probability $1$.
Otherwise, it will accept with probability $1/2$. 
Overall, the probability will be greater than $1/2$.

This shows that $\NP \subseteq \PP$.
\end{proof}
%%%%%%%

\begin{problem}
Let M be a probabilistic TM, and let $\Time(M,x)$ denote the number of steps taken by $M$ on input $x$.
Not that $\Time(M,x)$ is a {\it random variable} as well.

A probabilistic TM $M$ runs in {\bf strict polynomial-time} if there is a polynomial $p$ such that for all $x$, $\Prob[\Time(M,x) \leq p(|x|)] =1$. 
I.e., $M$ {\it always} takes at most $p(|x|)$ steps.

A probabilistic TM $M$ runs in {\bf expected polynomial-time} if there is a polynomial $p$ such that for all $x$, $\mathbb{E}[\Time(M,x)] \leq p(|x|)$.

The standard complexity classes $\PP$, $\BPP$, etc., are defined in terms of {\bf strict} polynomial time.
-
Define $\ePP$, to be the same as $\PP$, except we allow the \tm{} to be {\bf expected}-polynomial time.
That is, $L \in \ePP$ if there exists an expected-poly-time probabilistic TM $M$ with $L = \{ x \mid \Prob[M(x) = 1] > 1/2\}$.

Show that $\EXP \subseteq \ePP$.
In fact, it is possible to show that $\EXP$ can be decided (in the $\PP$ sense) by expected-{\it constant-time} probabilistic TMs.
\end{problem}

\begin{proof}

I will define the following \tm{} $M_{\ePP}$, where M is an $\EXP$-time \tm{}:

\fcodebox{
\underline{on input $(M,x,r)$:} \\
\> if $r$ contains only ones, then \\
\> \> simulate $M$ on $x$ \\
\> \> if $M$ accepts, accept \\
\> \> else reject \\
\> else \\
\> \> simulate $M$ for $p(|x|)$ steps: \\
\> \> if $M$ accepts, accept \\
\> \> if $M$ rejects, reject \\
\> \> if $M$ does not finish, accept w/ probability $1/2$
}

The \tm{} will take as input a random tape of size $p(|x|)$.
This way, the probability of it containing only the symbol $1$ is $\frac{1}{2^{p(|x|)}}$.
Let $r_1$ be this tape, and $n=|x|$.

The {\it expected running time} of this machine is:
%
\begin{align*}
	T(M_{\ePP}(M,x)) &= \frac{\Prob[r = r_i]*2^{p(n)} + \sum\limits_{r \in R \setminus \{r_1\}} p(n)}{|R|} \\
		&= \frac{\frac{1}{2^{p(n)}} 2^{p(n)} + 2^{p(n)-1}p(n)}{2^{p(n)}} \\
		&= \frac{1+2^{p(n)-1}p(n)}{2^{p(n)}} \\
		&\leq p(n)
\end{align*}
%
This shows that the expected-running-time of this machine is polynomial.
The probability that it accepts if $x \in L$ is:
%
\begin{align*}
	\Prob[M(x) =1] &= \Prob[r = r_i] + \frac{1}{2} \\
		&= \frac{1}{2^{p(n)}} + \frac{1}{2} \\
		&\geq \frac{1}{2}
\end{align*}
%
Therefore $M_{\ePP} \in \ePP$ and it can simulate any $M \in \EXP$.
This proves that $\EXP \subseteq \ePP$.

\end{proof}

%%%%%%%%%%%%

\begin{problem}
Let $L$ be a problem in $\RP$, so that:
\begin{align*}
	x \in L &\Rightarrow \Prob[M(x) = 1] \geq 1/2 \\
	x \not \in L &\Rightarrow \Prob[M(x) = 0] = 1
\end{align*}
Show that for each $n$, there exists a set $R$ of random tapes for $M$ with $|R| = n$ and satisfying the following property:
\[
	x \in L \Leftrightarrow \exists r \in R : M(x;r) = 1
\]
Show how this gives an alternate proof that $\RP \subseteq \Ppoly$.
\end{problem}

\begin{proof}
Similarly to the proof done in class for $\BPP \subseteq \Ppoly$, I will use Chernoff's bounds to amply the correctness of my \tm{}:
\begin{align*}
	x \in L &\Rightarrow \Prob[M(x)=1] \geq 1-2^{-|x|} \\
	x \not \in \L &\Rightarrow \Prob[M(x) = 0] = 1
\end{align*}
%
The probability that $r$ is bad for a particular $x$ is:
\[
	\Prob[r \mbox{ is bad for } x] < 2^{-n^c}
\]
%
Therefore, the probability of $r$ not existing is:
%
\begin{align*}
	\Prob [\forall r \in R : M(x;r) = 0] &= \Prob[r \mbox{ is bad for any } x] \cdot \Prob[r \in R] \\
		&= \sum_{x \in \{0,1\}^n} \Prob[r \mbox{ is bad for particular x }] \cdot \frac{n}{2^{n^c}} \\
		&= \frac{2^n}{2^{n^c}} \frac{n}{2^{n^{c-1}}} \\
		&= \frac{1}{2^{n^{c-1}}} \frac{n}{2^{n^{c-1}}} \\
		&= \frac{n}{2^{2n^{c-1}}} \\
		&< 1
\end{align*}
This shows that $\exists r \in R: M(x,r) = 1$.
This $r$ can be considered as a circuit that can be given to a $P$-time \tm{} as advice.
The maximum length of $r$ has to be polynomial, since the $\RP$ \tm{} must finish in polynomial time.
This proves that $\RP \subseteq \Ppoly$.
\end{proof}

\end{document}