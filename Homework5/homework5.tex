\documentclass[letterpaper,11pt]{article}

\usepackage[paper=letterpaper,margin=1in]{geometry}


%% minimal example of some latex stuff

\usepackage{xspace,graphicx,amsmath,amsthm,amssymb,xcolor}


\newtheorem{theorem}{Theorem}

%% code boxes

\usepackage{varwidth}
\newcommand{\fcodebox}[1]{%
    \framebox{\codebox{#1}}%
}
\newcommand{\codebox}[1]{%
        \begin{varwidth}{\linewidth}%
        \begin{tabbing}%
            ~~~\=\quad\=\quad\=\quad\=\quad\=\quad\=\kill
            #1
        \end{tabbing}%
        \end{varwidth}%
}

%% complexity classes

\newcommand{\cc}[1]{\ensuremath{\textsf{#1}}\xspace}
\renewcommand{\P}{\cc{P}}
\newcommand{\NP}{\cc{NP}}
\newcommand{\EXP}{\cc{EXP}}
\newcommand{\coNP}{\cc{coNP}}
\renewcommand{\L}{\cc{L}}
\newcommand{\NL}{\cc{NL}}
\newcommand{\PH}{\cc{PH}}
\newcommand{\PSPACE}{\cc{PSPACE}}
\newcommand{\Ppoly}{\cc{P/poly}}
\newcommand{\ZPP}{\cc{ZPP}}
\newcommand{\BPP}{\cc{BPP}}
\newcommand{\RP}{\cc{RP}}
\newcommand{\sharpP}{\cc{$\sharp$P}}
\newcommand{\AM}{\cc{AM}}
\newcommand{\MA}{\cc{MA}}
\newcommand{\IP}{\cc{IP}}


\newcommand{\DTIME}{\cc{DTIME}}
\newcommand{\NTIME}{\cc{NTIME}}
\newcommand{\DSPACE}{\cc{DSPACE}}
\newcommand{\NSPACE}{\cc{NSPACE}}
\newcommand{\SIZE}{\cc{SIZE}}


\newcommand{\karp}{\le_{\textsf{P}}}

%% decision problems

\newcommand{\prob}[1]{\ensuremath{\textsc{#1}}\xspace}
\newcommand{\SAT}{\prob{sat}}
\newcommand{\ThreeSAT}{\prob{3sat}}

%% homework templates

\newcounter{problem}
\newcounter{subproblem}[problem]

\newenvironment{problem}%
{%
	\stepcounter{problem}%
	\textbf{\theproblem.}
	\large
}{\\}%

\newenvironment{subproblem}%
{%
	\stepcounter{subproblem}%
	\textbf{\alph{subproblem})}
	\large
}{\\}%

%%%%

\newcommand{\tm}{Turing Machine}
\newcommand{\beq}{\begin{equation}}
\newcommand{\eeq}{\end{equation}}

%%%%

\title{Homework 5}
\author{Caius Brindescu}

\begin{document}

\maketitle

\begin{problem}
Prove that if $\P = \NP$ then $\P = \PH$.
\end{problem}

%%%%%%%%%%

\begin{proof}
%I know that $\P = \Sigma_0$ and $\NP = \Sigma_1$. 
%If $\P = \NP$, then $\Sigma_0 = \Sigma_1$.

%I can expand $\Sigma_0 = \{ x \mid M(x) = 1\}$ and $\Sigma_1 = \{ x \mid \exists w : M(x,w) = 1\}$.
%Using this I will show that $\Sigma_k = \Sigma_0$, where $\Sigma_k = \{ x \mid \exists w_1 \forall w_2 \ldots \square w_k : M(x,w_1,w_2,\ldots) = 1\}$.

$\P$ is closed under complement, so $\P = \NP$ implies that $\NP = \coNP$. 
Therefore, $\Sigma_1 = \Pi_1$. 
Using the proof done in class, I can deduce that $\Sigma_1 = \Pi_1 = \PH$.
However, $\P = \NP$, therefore $\Sigma_0 = \Sigma_1 = \PH$.
This gives us that $\P = \PH$.
\end{proof}

\begin{problem}
Let $\Delta = \NP \cap \coNP$. Prove that $\Delta = \P^\Delta$.
\end{problem}

%%%%%%%%%%%%

\begin{proof}
I know that that $\P \subset \Delta$, since $\P \subset \NP \cap \coNP$.

First I will prove that $\Delta \subseteq \P^\Delta$.
This is trivial, since my $\P^\Delta$ machine only has to query the oracle with input and return the oracle's response.

I will now have to prove that $\P^\Delta \subseteq \Delta$. 
This follows from the fact that $\P \subset \Delta$. 
If I have a machine $M \in \P^\Delta$, then any computation done by this machine (other than querying the oracle) is done in $\P$ time.
Since $\P \subset \Delta$, that means that a machine $M'$ can also do it, since it would only take polynomially more time than solving the $\Delta$ problem. 
This would keep $M' \in \Delta$.
This means that $\P^\Delta \subseteq \Delta$.

Therefore, $\P = \P^\Delta$.

\end{proof}

\begin{problem}
Prove that $\NP^{ \Sigma_k \cap \Pi_k } = \Sigma_k$ for all $k$
\end{problem}

%%%%%%%%%%

\begin{proof}

\end{proof}

\end{document}