\documentclass[letterpaper,11pt]{article}

\usepackage[paper=letterpaper,margin=1in]{geometry}


%% minimal example of some latex stuff

\usepackage{xspace,graphicx,amsmath,amsthm,amssymb,xcolor}

\newtheorem{theorem}{Theorem}

%% code boxes

\usepackage{varwidth}
\newcommand{\fcodebox}[1]{%
	\vspace{10pt}%
    \framebox{\codebox{#1}}%
    \vspace{10pt}%
}
\newcommand{\codebox}[1]{%
        \begin{varwidth}{\linewidth}%
        \begin{tabbing}%
            ~~~\=\quad\=\quad\=\quad\=\quad\=\quad\=\kill
            #1
        \end{tabbing}%
        \end{varwidth}%
}

%% complexity classes

\newcommand{\cc}[1]{\ensuremath{\textsf{#1}}\xspace}
\renewcommand{\P}{\cc{P}}
\newcommand{\NP}{\cc{NP}}
\newcommand{\EXP}{\cc{EXP}}
\newcommand{\coNP}{\cc{coNP}}
\renewcommand{\L}{\cc{L}}
\newcommand{\NL}{\cc{NL}}
\newcommand{\PH}{\cc{PH}}
\newcommand{\PSPACE}{\cc{PSPACE}}
\newcommand{\Ppoly}{\cc{P/poly}}
\newcommand{\ZPP}{\cc{ZPP}}
\newcommand{\BPP}{\cc{BPP}}
\newcommand{\RP}{\cc{RP}}
\newcommand{\sharpP}{\cc{$\sharp$P}}
\newcommand{\AM}{\cc{AM}}
\newcommand{\MA}{\cc{MA}}
\newcommand{\IP}{\cc{IP}}

\newcommand{\DTIME}{\cc{DTIME}}
\newcommand{\NTIME}{\cc{NTIME}}
\newcommand{\DSPACE}{\cc{DSPACE}}
\newcommand{\NSPACE}{\cc{NSPACE}}
\newcommand{\SIZE}{\cc{SIZE}}


\newcommand{\karp}{\le_{\textsf{P}}}

%% decision problems

\newcommand{\prob}[1]{\ensuremath{\textsc{#1}}\xspace}
\newcommand{\SAT}{\prob{sat}}
\newcommand{\ThreeSAT}{\prob{3sat}}

%% homework templates

\newcounter{problem}
\newcounter{subproblem}[problem]

\newenvironment{problem}%
{%
	\stepcounter{problem}%
	\textbf{\theproblem.}
	\large
}{\\}%

\newenvironment{subproblem}%
{%
	\stepcounter{subproblem}%
	\textbf{\alph{subproblem})}
	\large
}{\\}%

%%%%

\newcommand{\tm}{Turing Machine}
\newcommand{\beq}{\begin{equation}}
\newcommand{\eeq}{\end{equation}}

\title{Homework 1}
\author{Caius Brindescu}

\begin{document}

\maketitle

\begin{problem}
Consider the following variant of a single-tape Turing machine. 
At each step, it can either move one cell to the right or reset the tape-head to the leftmost cell of the tape. 
In particular, it cannot (natively) move just one cell to the left like a �standard� TM.

Show that this TM variant is equivalent in its computational power to a standard TM. 
In particular, this involves showing that the variant can simulate a standard TM, and vice-versa.
\end{problem}

\begin{proof}

The definition of the \tm{} variant is:
%
\beq
	Q \times \Gamma \leftarrow Q \times \Gamma \times \{L,0,1\}
\eeq
% 
where $L$ is the command to reset the tape to the leftmost cell of the tape.

I will prove that the \tm{} variant can simulate a regular \tm{}.

First, I will mark the current cell, each time I do a right shift:

\fcodebox{
\underline{moveTapeOneCellRight:} \\
\> move tape to the left most cell \\
\> scan until you find the mark \\
\> move right \textit{//this positions me on the original location} \\
\> mark the symbol on the tape \\
\> move right \textit{//the actual move right operation}
}
i
Then, I will use this algorithm to move the tape to the left:

\fcodebox{
\underline{moveTapeOneCellLeft:} \\
\> move tape to the left most cell \\
\> scan until I find the last mark \\
\> remove the mark
}

These two algorithms show that a standard \tm{} can be simulated by our variant.

Similarly, I will now prove that this machine can be simulated by a regular \tm{}.

To simulate moving to the end of the tape, we will simply move one stop to the left, until the end of the tape is reached.
To aid in the detection of the beginning of the tape, before the machine starts, I will make the first cell with a special symbol.
Then, returning to the beginning becomes:

\fcodebox{
\underline{rewindTape:} \\
\> scan tape until start mark is found
}

This shows that the variant is just as powerful as a regular \tm{}.
\end{proof}

\begin{problem}
Decidability
\end{problem}

\begin{subproblem}
Show that the following language is Turing-decidable:
\beq
	\small{L = \{ \langle M,x \rangle | M \text{ eventually changes a space character to a non-space character or input } x\}}
\eeq
\end{subproblem}

\begin{proof}

I am deriving the proof from the proof for Theorem 19, section 37.12 of the Jeff Erickson's Lecture 37.

Similarly, I will define the following \tm{}:

\fcodebox{
on input $\langle M,x \rangle$: \\
\> simulate M on x, one step at a time: \\
\> \> if M writes over a blank character, then accept \\
\> \> else if M accepts then reject \\
\> \> else increment $counter$ \textit{// stored on tape}\\
\> \> if $counter > C$ then reject
}

In the above algorithm, I define C as being the total number of configurations a \tm{} can have without using more tape than the size of the input.
Therefore:
\beq
	C = |Q| \times |\Gamma| \times |w|
\eeq

The \tm{} described above will always terminate.
If the \tm{} takes more steps than C, then I am repeating a configuration, and it enters an infinite loop.
Therefore, it will never overwrite a blank character, so I can safely reject that input.

Using this, I conclude that the proposed language is Turing-decidable
\end{proof}

\begin{subproblem}
Let \# be a special tape symbol (distinct from 0, 1, or the blank symbol).
Show that the following language is not Turing-decidable:
\beq
L = \{ \langle M, x \rangle | M \text{ eventually changes a space character to \# on input } x \}
\eeq
\end{subproblem}

\begin{proof}

Consider the following algorithm:

\fcodebox{
on input $\langle M, x \rangle$:\\
\> write down $M^{*}$: \\
\> $M^{*}$: \fcodebox{
	on input $x$: \\
	\> run $M$ on $x$ \\
	\> if $M$ accepts, then move right and write '\#' to the tape \\
	\> else write '1' to the tape
} \\
\> return $M^{*} \stackrel{?}{\in} L$
}

If we can decide if $M^{*} \in L$, then we can decide the acceptance problem.
Therefore: $ L_{acc} \leq_{T} L $, so $L$ is not Turing-decidable.

\end{proof}

\begin{problem}
For a language $L$, define $L^{*}=\{w|\exists w_{1},\ldots w_{k}:w=w_{1} \ldots w_{k} \text{ and each } w_{i} \in L\}$.\
\end{problem}
\begin{subproblem}
Prove that if $L$ is Turing decidable, then so is $L^{*}$.
\end{subproblem}

\begin{proof}
\end{proof}

\end{document}