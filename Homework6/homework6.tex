\documentclass[letterpaper,11pt]{article}

\usepackage[paper=letterpaper,margin=1in]{geometry}


%% minimal example of some latex stuff

\usepackage{xspace,graphicx,amsmath,amsthm,amssymb,xcolor}


\newtheorem{theorem}{Theorem}

%% code boxes

\usepackage{varwidth}
\newcommand{\fcodebox}[1]{%
    \framebox{\codebox{#1}}%
}
\newcommand{\codebox}[1]{%
        \begin{varwidth}{\linewidth}%
        \begin{tabbing}%
            ~~~\=\quad\=\quad\=\quad\=\quad\=\quad\=\kill
            #1
        \end{tabbing}%
        \end{varwidth}%
}

%% complexity classes

\newcommand{\cc}[1]{\ensuremath{\textsf{#1}}\xspace}
\renewcommand{\P}{\cc{P}}
\newcommand{\NP}{\cc{NP}}
\newcommand{\EXP}{\cc{EXP}}
\newcommand{\coNP}{\cc{coNP}}
\renewcommand{\L}{\cc{L}}
\newcommand{\NL}{\cc{NL}}
\newcommand{\PH}{\cc{PH}}
\newcommand{\PSPACE}{\cc{PSPACE}}
\newcommand{\Ppoly}{\cc{P/poly}}
\newcommand{\ZPP}{\cc{ZPP}}
\newcommand{\BPP}{\cc{BPP}}
\newcommand{\RP}{\cc{RP}}
\newcommand{\sharpP}{\cc{$\sharp$P}}
\newcommand{\AM}{\cc{AM}}
\newcommand{\MA}{\cc{MA}}
\newcommand{\IP}{\cc{IP}}


\newcommand{\DTIME}{\cc{DTIME}}
\newcommand{\NTIME}{\cc{NTIME}}
\newcommand{\DSPACE}{\cc{DSPACE}}
\newcommand{\NSPACE}{\cc{NSPACE}}
\newcommand{\SIZE}{\cc{SIZE}}


\newcommand{\karp}{\le_{\textsf{P}}}
\newcommand{\logred}{\le_{\textsf{L}}}

%% decision problems

\newcommand{\prob}[1]{\ensuremath{\textsc{#1}}\xspace}
\newcommand{\SAT}{\prob{sat}}
\newcommand{\ThreeSAT}{\prob{3sat}}

%% homework templates

\newcounter{problem}
\newcounter{subproblem}[problem]

\newenvironment{problem}%
{%
	\stepcounter{problem}%
	\textbf{\theproblem.}
	\large
}{\\}%

\newenvironment{subproblem}%
{%
	\stepcounter{subproblem}%
	\textbf{\alph{subproblem})}
	\large
}{\\}%

%%%%

\newcommand{\tm}{Turing Machine}
\newcommand{\beq}{\begin{equation}}
\newcommand{\eeq}{\end{equation}}

%%%%

\title{Homework 6}
\author{Caius Brindescu}

\begin{document}

\maketitle

\begin{problem}
An oracle TM $M$ is {\bf downward-querying} if on input of length $n$, every query to the oracle has length strictly less than $n$. 
A decision problem $X$ is {\bf downward self-reducible} if $X = L(M^X)$ for some downward-querying, polynomial-time oracle TM $M$. 
Note that $M$ has oracle access to the problem that it is trying to solve, but when trying to determine whether $x \in X$, the TM cannot simply query its oracle on $x$ for the answer.
\end{problem}

\begin{subproblem}
Show that TQBF is downward self-reducible. 
What assumptions are you making about how formulas are encoded into bit strings?
\end{subproblem}

\begin{proof}
\[
	TQBF = \{\phi \mid \exists x_1, \forall x_2, \ldots \square x_k: \phi(x_1,x2,\ldots x_k) = 1\}
\]
I am assuming that the input tape first contains the representation of the boolean variables, as a set of strings.
After that, we have the formula, written in terms of those variables and boolean operators. 

Let $M^{TQBF}$ be a \tm{} that can decide $TQBF$.
It is downward querying. 
I define it as follows:

\fcodebox{
\underline{$M^X$ on input $x$:} \\
\> nondeterministically choose first variable, $x_1$. \\
\> query the oracle $X$ with the variable $x_1$ set \\
\> if oracle accepts, accept \\
\> else reject
}

$M \in \NP$, therefore it runs in polynomial time.
Therefore, $TQBF$ is downward self-reducible.
\end{proof}

%%%%%%%%%%%%%%%%

\begin{subproblem}
Prove that if $L$ is downward self-reducible then $L \in PSPACE$.
\end{subproblem}

\begin{proof}
According to the definition, $L = L(M^L)$, where $M$ is a polynomial time \tm{}.
Trivially, $M \in \PSPACE$, because $\P \subset \PSPACE$.

Because L is downward self-reducible, I know that it makes queries strictly shorter than it's input.
If we chose to replace the oracle with a \tm{} $M'^L$, then $M'$ will also run in polynomial time.
Since $M$ runs in polynomial time it can, at most, do a polynomial number of queries to the oracle.
Therefore, it also runs in polynomial time.

I can do this recursively, until I inline all the oracles, and I will still get a \tm{} that runs in polynomial time.
Therefore, I can conclude that $L \in PSPACE$.
\end{proof}

%%%%%%%%%%%%%%%%%%

\begin{problem}
Prove that logspace reductions are transitive. 
That is, if $A \logred B$ and $B \logred C$ then $A \logred C$.
\end{problem}

\begin{proof}
The fact that $A \logred B$, then $\exists f : x \in A \Rightarrow f(x) \in B$.
And, if $B \logred C$, then $\exists g : x \in A \Rightarrow g(x) \in C$.
Then, $A \logred C$ can be written as $\exists h: x \in A \Rightarrow h(x) \in C \mbox{ and } h = f \circ g$.

First of all, the traditional way of proving transitivity does not work for logspace. 
Let $M_f$ be a \tm{} that computes $f$, and $M_g$ a \tm{} that computes $g$.
Both $M_f$ and $M_g$ can run in logspace. 
I cannot compute $h(x)$, by computing the output of $M_g$ on input $x$ and then using it's output as input for $f$.
The output of $M_g$ might not fit in logarithmic space.

To prove this, I will have to give the input for $M_B$ one at a time.
I will define the following \tm{} $M_h$:

\fcodebox{
\underline{on input $M_f$, $M_g$, $x$:} \\
\> run $M_g$ until it asks for a cell $i$ of the input tape \\
\> suspend $M_g$ \\
\> run $M_f$ on input $x$ until it outputs cell $i$ of the input tape \\
\> resume $M_g$ with the calculated input \\
\> repeat until $M_g$ accepts, then accept
}

The input for $M_h$ is given to $M_f$ and the output tape is the same as the output tape of $M_g$.

$M_h$ requires the space for running $M_f$ and $M_g$ and the space needed to store the cell $i$ that $M_g$ wants to read and space to store the actual output of $M_f$ for that cell.
This means that $M_h$ runs in logspace.
\end{proof}

%%%%%%%%%%%%%%

\begin{problem}
Show that $\P/1$ (i.e., languages computable by polynomial-time TMs with 1 bit of advice) contains undecidable problems.
\end{problem}

\begin{proof}
Such a machine could solve the halting problem.
Let the input be the encoding a \tm{} $M$ and the advice be if it halts or not.
\end{proof}

\end{document}