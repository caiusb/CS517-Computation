\documentclass[letterpaper,11pt]{article}

\usepackage[paper=letterpaper,margin=1in]{geometry}


%% minimal example of some latex stuff

\usepackage{xspace,graphicx,amsmath,amsthm,amssymb,xcolor}


\newtheorem{theorem}{Theorem}

%% code boxes

\usepackage{varwidth}
\newcommand{\fcodebox}[1]{%
    \framebox{\codebox{#1}}%
}
\newcommand{\codebox}[1]{%
        \begin{varwidth}{\linewidth}%
        \begin{tabbing}%
            ~~~\=\quad\=\quad\=\quad\=\quad\=\quad\=\kill
            #1
        \end{tabbing}%
        \end{varwidth}%
}

%% complexity classes

\newcommand{\cc}[1]{\ensuremath{\textsf{#1}}\xspace}
\renewcommand{\P}{\cc{P}}
\newcommand{\NP}{\cc{NP}}
\newcommand{\EXP}{\cc{EXP}}
\newcommand{\coNP}{\cc{coNP}}
\renewcommand{\L}{\cc{L}}
\newcommand{\NL}{\cc{NL}}
\newcommand{\PH}{\cc{PH}}
\newcommand{\PSPACE}{\cc{PSPACE}}
\newcommand{\Ppoly}{\cc{P/poly}}
\newcommand{\ZPP}{\cc{ZPP}}
\newcommand{\BPP}{\cc{BPP}}
\newcommand{\RP}{\cc{RP}}
\newcommand{\sharpP}{\cc{$\sharp$P}}
\newcommand{\AM}{\cc{AM}}
\newcommand{\MA}{\cc{MA}}
\newcommand{\IP}{\cc{IP}}


\newcommand{\DTIME}{\cc{DTIME}}
\newcommand{\NTIME}{\cc{NTIME}}
\newcommand{\DSPACE}{\cc{DSPACE}}
\newcommand{\NSPACE}{\cc{NSPACE}}
\newcommand{\SIZE}{\cc{SIZE}}


\newcommand{\karp}{\le_{\textsf{P}}}

%% decision problems

\newcommand{\prob}[1]{\ensuremath{\textsc{#1}}\xspace}
\newcommand{\SAT}{\prob{sat}}
\newcommand{\ThreeSAT}{\prob{3sat}}

%% homework templates

\newcounter{problem}
\newcounter{subproblem}[problem]

\newenvironment{problem}%
{%
	\stepcounter{problem}%
	\textbf{\theproblem.}
	\large
}{\\}%

\newenvironment{subproblem}%
{%
	\stepcounter{subproblem}%
	\textbf{\alph{subproblem})}
	\large
}{\\}%

%%%%

\newcommand{\tm}{Turing Machine}
\newcommand{\beq}{\begin{equation}}
\newcommand{\eeq}{\end{equation}}

%%%%

\begin{document}

\section*{LaTeX examples}

\begin{theorem}[Ladner]
    If $\P \ne \NP$ then there exists problems in $\NP$ which are neither \NP-complete nor in \P.
\end{theorem}
\begin{proof}
    We will demonstrate two problems $L_0$ and $L_1$ satisfying:
    \begin{itemize}
        \item $L_0, L_1 \in \NP \setminus \P$
        \item $L_0 \not\karp L_1$
        \item $L_1 \not\karp L_0$
    \end{itemize}
    This will imply that $L_0$ is not \NP-complete, since in particular the \NP language $L_1$ does not Karp-reduce to it.

The idea is to design $L_0$ and $L_1$ so that $L_0 \cap L_1 = \emptyset$ but $L_0 \cup L_1 = \ThreeSAT$. Hence, we will ``fill up'' $L_1$ with \ThreeSAT for some time, then switch to ``filling up'' $L_1$ with \ThreeSAT, and so on. 

How will we choose when to ``switch'' between $L_0$ and $L_1$? By trying to diagonalize against all candidate Karp reductions from $L_0$ to $L_1$. 

Let $M_1, M_2, \ldots$ be an enumeration of polynomial-time TMs (with output), with corresponding polynomial time bounds $p_1, p_2, \ldots$. Then our two languages $L_0$ and $L_1$ above are constructed via $L_b = L(M^*_b)$, where:
\begin{center}
    \fcodebox{
        \underline{$M^*_0(x)$:} \\
        \> $b := 0$ \\
        \> for a total of $|x|$ steps: \\
        \> \> for $i = 1, 2, \ldots$: \\
        \> \> \> for $j = 1, 2, \ldots$: \\
        \> \> \> \> {\sl // compute via exponential-} \\
        \> \> \> \> {\sl // time algorithm:} \\
        \> \> \> \> $\alpha := [ \langle j \rangle \overset?\in L_{b}]$ \\
        \> \> \> \> $\beta := [ M_i(\langle j\rangle ) \overset?\in L_{1-b}]$ \\
        \> \> \> \> if $\alpha \ne \beta$: \\
        \> \> \> \> \> $b := 1 - b$ \\
        \> \> \> \> \> next $i$ \\
        \> if \underline{$b=0$}: \\
        \> \> run \NP-algo for $x \overset?\in \ThreeSAT$ \\
        \> else return 0
    }
    \qquad
    \fcodebox{
        \underline{$M^*_1(x)$:} \\
        \> $b := 0$ \\
        \> for a total of $|x|$ steps: \\
        \> \> for $i = 1, 2, \ldots$: \\
        \> \> \> for $j = 1, 2, \ldots$: \\
        \> \> \> \> {\sl // compute via exponential-} \\
        \> \> \> \> {\sl // time algorithm:} \\
        \> \> \> \> $\alpha := [ \langle j \rangle \overset?\in L_{b}]$ \\
        \> \> \> \> $\beta := [ M_i(\langle j\rangle ) \overset?\in L_{1-b}]$ \\
        \> \> \> \> if $\alpha \ne \beta$: \\
        \> \> \> \> \> $b := 1 - b$ \\
        \> \> \> \> \> next $i$ \\
        \> if \underline{$b=1$}: \\
        \> \> run \NP-algo for $x \overset?\in \ThreeSAT$ \\
        \> else return 0
    }
\end{center}
It now occurs to me that this was only meant to be a simple example of formatting macros. So maybe I should stop so as to not spoil the surprise.
\end{proof}

\end{document}